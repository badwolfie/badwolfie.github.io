\documentclass[12pt, letterpaper, oneside]{report}
\usepackage[T1]{fontenc}
\usepackage[utf8]{inputenc}
\usepackage[spanish]{babel}
\usepackage{amsmath,amsfonts,amsthm}
\usepackage{titlesec}
\usepackage{multirow}
\usepackage{graphicx}
\usepackage{ragged2e}
\usepackage{xspace}

\usepackage[papername={letterpaper},top=2cm,bottom=2cm,left=2.5cm,right=2.5cm]{geometry}
\setlength\parindent{0pt}

\pagestyle{empty}
\begin{document}
	\centering\Large\textbf{Ian Yevgeni Hernández Sánchez} \\[7mm]
	\justify\normalsize
	\begin{tabular}{l l}
	\textbf{Domicilio:} & Belice 39, Col. Olivar de los Padres, Álvaro Obregón, \\
	 & Distrito Federal, C.P. 01780 \\[2mm]
	\textbf{Correo Electrónico:} & ihernandezs@openmailbox.org \\[2mm]
	\textbf{Teléfono:} & (55) 45 90 33 88 \\[7mm]
	\end{tabular}

	\section{Experiencia Profesional}
	\justify
	\textbf{Prestador de Servicio Social}, Sep. 2014 - Abr. 2015 \\
	Centro de Investigación en Computación del Instituto Politécnico Nacional 
	(CIC-IPN). \\[2mm]
	Desarrollo de módulos y mantenimiento del Sistema de Administración de la 
	Base de datos Escolar (SABER) del CIC-IPN y otras labores de apoyo en el 
	Departamento de Integración Tecnológica. \\[5mm]
	\textbf{Programador Jr}, Mayo 2015 - Actualidad \\
	Centro de Investigación en Computación del Instituto Politécnico Nacional 
	(CIC-IPN). \\[5mm]

	\centering\textbf{Preparación Académica} \\
	\justify
	\textbf{Ingeniería en Sistemas Computacionales}, 2011 - Actualidad \\
	Escuela Superior de Cómputo del Instituto Politécnico Nacional 
	(ESCOM-IPN). \\[5mm]
	\textbf{Idiomas: Inglés (hablado y escrito)} \\[5mm]
	\textbf{Cursos:}
	\begin{itemize}
		\item \textbf{Curso de Inglés Cambridge ESOL, 9 niveles completados.} \\
		Examen PET 89/100 puntos.
	\end{itemize}
	\textbf{Competencias y Habilidades:} 
	Desarrollo de software, manejo de diversos lenguajes de programación 
	(principalmente C y Java), manejo de proyectos con Git, administración de 
	bases de datos en MySQL, experiencia en el manejo de sistemas operativos 
	tipo UNIX (principalmente distribuciones GNU/Linux), rápido aprendizaje, 
	gusto por la resolución de problemas y curiosidad por aprender cosas nuevas. \\[5mm]
	\textbf{Áreas de interés:} Desarrollo de software, Programación en C, 
	Admninistración de sistemas tipo UNIX, Criptografía.
\end{document}